\documentclass[conference]{IEEEtran}
\IEEEoverridecommandlockouts
% The preceding line is only needed to identify funding in the first footnote. If that is unneeded, please comment it out.
\usepackage{cite}
\usepackage{amsmath,amssymb,amsfonts}
\usepackage{algorithmic}
\usepackage{graphicx}
\usepackage{textcomp}
\usepackage{xcolor}
\def\BibTeX{{\rm B\kern-.05em{\sc i\kern-.025em b}\kern-.08em
    T\kern-.1667em\lower.7ex\hbox{E}\kern-.125emX}}
\begin{document}

\title{A theoretical analysis of the Keep Network random beacon using agent based modeling \\
\thanks{Identify applicable funding agency here. If none, delete this.}
}

\author{\IEEEauthorblockN{1\textsuperscript{st} Prashanth Irudayaraj}
\IEEEauthorblockA{\textit{Keep Network} \\
prashanth@keep.network}

\and
\IEEEauthorblockN{2\textsuperscript{nd} Antonio Cordozo}
\IEEEauthorblockA{\textit{Keep Network} \\
antonio@keep.network}

\and
\IEEEauthorblockN{4\textsuperscript{th} Promethea Raschke}
\IEEEauthorblockA{\textit{Keep Network} \\
promethea@keep.network}

\and
\IEEEauthorblockN{5\textsuperscript{th} Markus Fix}
\IEEEauthorblockA{\textit{Keep Network} \\
markus@keep.network}

\and
\IEEEauthorblockN{6\textsuperscript{th} Liam Zebedee}
\IEEEauthorblockA{\textit{Keep Network} \\
liam@keep.network}
}

\maketitle

\begin{abstract}
    ABM is an effective tool for analysis of complex systems with long term 
    emergent behavior. In this study we apply ABM to analyse the long term 
    behavior of Keep’s random beacon. We look for the emergence of steady 
    state behavior, at which point we evaluate the sensitivity of specific 
    group and signature characteristics to various parameters. The results 
    of this study illustrates the effectiveness of ABM as an aid to the design 
    of novel distributed systems. 

\end{abstract}

\begin{IEEEkeywords}
Token Engineering, Agent Based Model, Distributed Systems
\end{IEEEkeywords}

\section{Introduction}

\section{Agent Based Modeling (ABM)}
ABM has traditionally been a tool to simulate complex dynamic systems 
such as the spread of pathogens \cite{Bauer2009}, social psychology 
\cite{Smith2007}, and financial markets \cite{Feng2012}. It is well suited
to systems that resist simple analytical solutions due to the interaction of 
complex individual agents with varying attributes. As a bottom up approach, ABM
has been gaining in popularity over tradtional methods such as Discrete Event
Simulation, and can provide more convincing theoretical analysis than approaches
such as general equilibrium analysis \cite{Wang2018}.

\section{Analysis of Token based systems}


Discuss the use of ABM for token engineering. Cite recent papers.

\section{Overview of the Keep Random Beacon}

\subsection{Role of the model in the design process}

\subsection{Key research questions}

\section{Model Creation}

\subsection{Terminology}

\subsection{Structure of the Model}
\begin{itemize}

\item Single run
\item Multiple runs
    
\end{itemize}

\subsection{Assumptions}
\begin{itemize}
\item Stochastic Assumptions
\item Ticket distribution Assumptions
\item Model Assumptions
\end{itemize}
    
\section{Verification and Validation}
Benchmark using analytical methods - from Promethea

\section{Results}
\begin{itemize}
\item Emergence of steady state behavior
\item Effects of Node failures on group and signature characteristics
\item Effects of stake distributions on group and signature characteristics
\item Additional analyses?
\end{itemize}

\section{Conclusion and Future work}
\begin{itemize}
\item Dynamic systems -Transfer functions instead of stochastic representations?
\item Utility functions
\item Genetic Algorithms
\item Additional questions
\end{itemize}

\bibliographystyle{plain}
\bibliography{Simulation_paper.bib}
\end{document}
