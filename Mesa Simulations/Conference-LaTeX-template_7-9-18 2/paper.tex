\documentclass[conference]{IEEEtran}
\IEEEoverridecommandlockouts
% The preceding line is only needed to identify funding in the first footnote. If that is unneeded, please comment it out.
\usepackage{cite}
\usepackage{amsmath,amssymb,amsfonts}
\usepackage{algorithmic}
\usepackage{graphicx}
\usepackage{textcomp}
\usepackage{xcolor}
\def\BibTeX{{\rm B\kern-.05em{\sc i\kern-.025em b}\kern-.08em
    T\kern-.1667em\lower.7ex\hbox{E}\kern-.125emX}}
\begin{document}

\title{A theoretical analysis of the Keep Network random beacon using agent based modeling \\
\thanks{Identify applicable funding agency here. If none, delete this.}
}

\author{\IEEEauthorblockN{Prashanth Irudayaraj}
\IEEEauthorblockA{\textit{Keep Network} \\
prashanth@keep.network}

\and
\IEEEauthorblockN{Antonio Cordozo}
\IEEEauthorblockA{\textit{Keep Network} \\
antonio@keep.network}

\and
\IEEEauthorblockN{Promethea Raschke}
\IEEEauthorblockA{\textit{Keep Network} \\
promethea@keep.network}

\and
\IEEEauthorblockN{Markus Fix}
\IEEEauthorblockA{\textit{Keep Network} \\
markus@keep.network}

\and
\IEEEauthorblockN{Liam Zebedee}
\IEEEauthorblockA{\textit{Keep Network} \\
liam@keep.network}
}

\maketitle

\begin{abstract}
    ABM is an effective tool for analysis of complex systems with long term 
    emergent behavior. In this study we apply ABM to analyse the long term 
    behavior of Keep’s random beacon. We look for the emergence of steady 
    state behavior, at which point we evaluate the sensitivity of specific 
    group and signature characteristics to various parameters. The results 
    of this study illustrates the effectiveness of ABM as an aid to the design 
    of novel distributed systems. 

\end{abstract}

\begin{IEEEkeywords}
Token Engineering, Agent Based Model, Distributed Systems
\end{IEEEkeywords}

\section{Introduction}

\section{Agent Based Modeling (ABM)}
ABM has traditionally been a tool to simulate complex dynamic systems 
such as the spread of pathogens \cite{Bauer2009}, social psychology 
\cite{Smith2007}, and financial markets \cite{Feng2012}. It is well suited
to systems that resist simple analytical solutions due to the interaction of 
complex individual agents with varying attributes. As a bottom up approach, ABM
has been gaining in popularity over tradtional methods such as Discrete Event
Simulation, and can provide more convincing theoretical analysis than approaches
such as general equilibrium analysis \cite{Wang2018}.

The nascent field of token engineering is currently establishing its own set 
of tools and processes. As such, ABM lends itself well to analyzing these systems
to support the design process and evaluate system state after launch.

\section{Analysis of Token based systems}


Discuss the use of ABM for token engineering. Cite recent papers.

\section{Overview of the Keep Random Beacon}
The Keep Network random beacon uses a threshold relay simuliar to the one proposed
by DFINITY and based on the BLS signature scheme \cite{} (NEED CITATION)

\subsection{Role of the model in the design process}
The initial architecture of the Keep beacon was designed by (need input from Antonio)
what purpose did the simulation serve?

\section{Key research questions}
\subsection{When could steady state behavior emerge?}
Emergence of steady state behavior occurs as a confluence of several factors and is difficult
to predict at the design stage. Specifically, process that take several blocks to complete 
such as DKG that often occur asynchronously can create large swings in values such as number of active
groups or percentage of compromised groups. 

\subsection{What are the effects of node failures on system behavior?}
Nodes may go offline or fail entireley. Since their participation in groups is critical to the 
success of the threshold relay, it is important to understand the impact of various levels
of failure on group characteristics. In particular, failures can result in disproportionate ownership
of groups which may cause a byzantine fault to occur. 

\subsection{What are the effects of stake distributions on system behavior?}
Stake distributions can also skew ownership of groups and can lead to a byzantine fault. A more
centralized distribution could once again lead to a greater ownership of groups by a few entities
thus creating conditions for the group to be compromised. 

\section{Model Creation}

\subsection{Model Structure}
We construct the model using the MESA ABM framework (cite). MESA consists of an agent class with
attributes and methods. We generate 3 types of agents Nodes, Groups, and Signatures. TABLE shows 
the specific differences in each of these.

ADD table for Agent types

The simulation model consists of the model class which instantiates the various agents
and steps through the simulation. At each step the state of the model and each agent
is updated. The scheduler manages the sequence of state changes. For this model we use
a simulataneous activation scheduler, which first stages updates for each agent and then 
advances them simultaneously.  (NEED TO CHECK SCHEDULER SEQUENCE AGAIN)

\subsection{Terminology}
Node: A computational entity with memory and computational power sufficient to to run a 
keep client

Node Owner: may own 1 or more nodes and can allocate different levels of stake to each node
owned

Stake: A bond that make a node eligible to participate in a group 

Group: A collection of nodes who have successfully completed DKG together

DKG (Distributed Key Generation): The processes of generating keys for each node enabling
them to sign using Keep's version of Threshold ECDSA \cite{Gennaro2018}.

Signature: The process of securely generating a random number using Threshold ECDSA \cite{Gennaro2018}

Ownership Percent: The level of ownership a specific entity has in a group or signature

Lynchpin: A node who's ownership exceeds the maximum malicious threshold

\subsection{Runs}
\begin{itemize}

\item Single run
\item Multiple runs
    
\end{itemize}

\subsection{Assumptions}
\begin{itemize}
\item Stochastic Assumptions
\item Ticket distribution Assumptions
\item Model Assumptions
\end{itemize}
    
\section{Verification and Validation}
Benchmark using analytical methods - from Promethea

\section{Results}
\begin{itemize}
\item Emergence of steady state behavior
\item Effects of Node failures on group and signature characteristics
\item Effects of stake distributions on group and signature characteristics
\item Additional analyses?
\end{itemize}

\section{Conclusion and Future work}
\begin{itemize}
\item Dynamic systems -Transfer functions instead of stochastic representations?
\item Utility functions
\item Genetic Algorithms
\item Additional questions
\end{itemize}

\bibliographystyle{plain}
\bibliography{Simulation_paper.bib}
\end{document}
